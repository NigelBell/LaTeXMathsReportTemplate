\documentclass{article}
\usepackage{lastpage} % number of last page 
\usepackage[margin=1in,headheight=13.6pt]{geometry}
\usepackage{lipsum}
\usepackage{fancyhdr}

% Set the headings' appearance in the ``fancy'' pagestyle
\pagestyle{fancy}
\fancyhf{}
\fancyhead[RO, LE]{\scshape\nouppercase{\leftmark}}
\fancyfoot[RO, LE]{\thepage}

% The first pages shall be empty, even no page numbering
\fancypagestyle{plain}{%
  \renewcommand{\headrulewidth}{0.0pt}%
  \fancyhf{}%
}
	
\renewcommand\footrule{\begin{minipage}{1\textwidth}
\hrule width \hsize height 2pt
\end{minipage}\par}%

\renewcommand\headrule{
\begin{minipage}{1\textwidth}
\hrule width \hsize height 2pt 
\end{minipage}}%
	
\lhead{courseName - assignmentName}
\rhead{name (username)}
%\lfoot{\today}
\rfoot{Page \thepage\ of \pageref{LastPage}}

\begin{document}
	\begin{enumerate}
		\item Answer to question 1:
			% ========== Just examples, please delete before submitting
			\begin{itemize}
				\item Use inline equations for simple math $1+1=2$, and centered equations for more involved or important equations
					\begin{equation}
					    a^2 + b^2 = c^2.
					\end{equation}
				\item Some people like to write scalars without boldface $x+y=1$ and vectors or matrices in boldface
					\begin{equation}
					    \mathbf{A} \mathbf{x} = \mathbf{b}.
					\end{equation}
				\item An example of a matrix:
					\begin{equation}
					    \mathbf{A} = \left(
					    \begin{array}{ccc}
						    3 & -1 & 2 \\ 	
						    0 & 1 & 2 \\ 
						    1 & 0 & -1 \\
						\end{array} 
						\right).  
					\end{equation}
				\item With a labeled equation such as the following:
					\begin{equation}
					    \label{accel}
					    \frac{d^2 x}{d t^2} = a
					\end{equation}
					you can refer to the equation later. In equation \ref{accel} we defined acceleration.
			\end{itemize}
		\item Answer to question 2:
			% ========== Just examples, please delete before submitting
			\begin{enumerate}
				\item Let $x = 3$
				\begin{enumerate}
					\item Equation 1: $1 + x = 4$
					\item Equation 2: $2 + x = 5$
					\item Equation 3: $3 + x = 6$
				\end{enumerate}
			\end{enumerate}
		\item Answer to question 3:
		
		\item Answer to question 4:
		
		\item Answer to question 5:
		
		\item Answer to question 6:
		
		\item Answer to question 7:
		
	\end{enumerate}
\end{document}